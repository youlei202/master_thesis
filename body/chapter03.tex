\chapter{基于移动模式最优节点群组选取的路由算法}

基于节点的移动模式,本章提出构造最优节点群组作为中继节点群的路由算法。在许多社会容迟网络(Social Delay Tolerant Networks, SDTN)中,具有共同兴趣的移动用户往往访问一些与其兴趣相关的地点。研究表明,50\%的移动用户会在某一个特定的接入点(access point, AP)上花费约74\%的时间\cite{Henderson:2004ul}。换言之,节点往往具有频繁访问某一或某一部分地点(简称为常访地点)的特点。这些常访地点可被看做``连接''这些节点的枢纽。可以通过在常访地点部署缓存设备,用以辅助消息传递,例如投掷盒(throw-box)\cite{Ibrahim:2009we}等设备。缓存设备具有普通移动节点不具备的优势。首先,由于部署的缓存设备位置固定在常访地点,且节点往往在常访地点停留一段时间,所以节点与缓存设备之间具有较为稳定的连接。不同与此,

\section{系统模型及基本定义}

\subsection{网络模型}

\subsection{基本定义}

\section{路由问题概览}

\subsection{移动模式定义}

\subsection{路由相关的两个关键属性}

\subsubsection{投递概率}

\subsubsection{期望时延}

\subsection{路由问题形式化定义}

\section{$N_{opt}$搜索问题分析}

\subsection{计算复杂性证明}

\subsection{局部陷阱}

\subsection{基于禁忌搜索的求解方法}

\section{移动模式相关的最优化路由}

\subsection{Local-MPAR:基于局部搜索的路由算法}

\subsection{Tabu-MPAR:基于禁忌搜索的路由算法}

\section{仿真实验}

\subsection{自变量:消息生存时间}

\subsection{自变量:节点缓存}

\section{结论}






