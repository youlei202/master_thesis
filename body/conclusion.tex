%%==================================================
%% conclusion.tex for SJTU Master Thesis
%% based on CASthesis
%% modified by wei.jianwen@gmail.com
%% version: 0.3a
%% Encoding: UTF-8
%% last update: Dec 5th, 2010
%%==================================================

\chapter{总结及展望}
%\chapter*{总结及展望\markboth{总结及展望}{}}
%\addcontentsline{toc}{chapter}{全文总结}

目前,针对容迟网络的研究主要集中在路由协议领域,如何做出正确高效的路由选择是无线网络领域内的关键技术和主要研究课题。通过对近年来的主要研究成果进行分析,移动社会网络报文传输机制的研究是容迟网络研究在引入社会网络分析等技术后的最新发展趋势。最新的移动社会网络信息共享研究也是强调在容迟网络分布式体系结构下高容量、低成本的数据传输。本文主要工作总结如下。

~\\
\noindent\textbf{1. 基于移动模式的最优节点群组选取算法} 

本文基于移动模式最优节点群组选取算法, \textbf{建立了周期相关的移动记录模型,并从移动记录中提取出节点(群组)移动模式}。对于成组的节点,其被看做一个整体,并评估整体的预测投递率。本文对两个相关路由的关键属性进行研究分析,并\textbf{将路由问题建模为组合最优化问题,且证明了该问题的$\mathcal{NP}$难解性}。为求解该路由问题,基于局部搜索及禁忌搜索,分别\textbf{提出了两种路由算法Local-MPAR及Tabu-MPAR}。此外,\textbf{证明了Tabu-MPAR过程可以使得持有消息的节点集合最终达到预测投递概率最优的集合}。仿真实验表明:两种MPAR算法,在机会网络中的综合移动模型WDM上优于DF算法及SimBet算法。

~\\
\noindent\textbf{2. 基于消息传输收益的最优队列调度算法}

本文基于消息传输收益的最优队列调度算法,针对节点间带宽和接触时间受限所导致的连接有限的吞吐量会造成消息的传输失败,从而浪费了节点间宝贵的接触机会的现象,通过\textbf{定义概念“消息传输效用”,在PRoPHET的基础上加上数据项选择机制,得到改进后的路由算法Throughput}。仿真实验表明:该数据项选择机制能够在消息投递率,消息投递时延,网络开销三方面较大程度的改进路由性能表现。

~\\
\noindent\textbf{3. 基于跳数的启发式距离向量算法}

最后,本文利用启发函数,基于跳数信息进行消息投递跳数预测。利用滑动窗口机制,可以动态更新记录平均跳数的矩阵。基于此,\textbf{定义了启发式函数用于预测当前节点到目的节点之间的潜在跳数}。为了便于计算,定义了一种矩阵运算符,从而\textbf{将跳数的估计过程转化为矩阵运算}。仿真实验表明:本文提出的HCH算法在综合性能表现上优于Epidemic, S \& W以及PRoPHET三种算法。

~\\

智能交通系统所依赖的一种组网方式,即车联网,亦具有延迟容忍的特性,最新的研究趋势则是从延迟容忍网络及移动社会网络的角度,对车联网的路由问题进行研究。本文提出的基于移动模式的节点群组选取算法及对应的路由协议,正可用于社会属性相关的一类网络中。未来的研究将可集中于车联网的环境,针对车联网中既具有社会属性亦具有周期移动特性的一类节点进行分析,建立出能够如实反应其移动模式的数学模型,从而实现最优路由。此外,消息副本的分发速度是衡量路由性能的很重要的指标,本文对网络中消息副本分发速度只从仿真实验进行了验证,而未从理论上保证其可靠性。在以后的工作中,拟建立数学模型,对消息副本的分发过程进行分析,从理论上推导消息副本数量与分发时延之间的关系,从而保证消息的投递时延。

