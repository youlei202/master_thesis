%%==================================================
%% conclusion.tex for SJTU Master Thesis
%% based on CASthesis
%% modified by wei.jianwen@gmail.com
%% version: 0.3a
%% Encoding: UTF-8
%% last update: Dec 5th, 2010
%%==================================================

\chapter*{全文总结\markboth{全文总结}{}}
\addcontentsline{toc}{chapter}{全文总结}

近年来,针对容迟网络的研究主要集中在路由协 议领域,如何做出正确高效的路由选择是无线网络领域内的关键技术和主要研究课题。本文的研究点共分为三部分。

第一部分为基于移动模式最优节点群组选取算法,建立了周期相关的移动记录模型,并从移动记录中提取出节点(群组)移动模式。对于成组的节点,其被看做一个整体,并评估整体的预测投递率。本章研究了两个相关路由的关键属性,并将路由问题建模为组合最优化问题,且证明了该问题的$\mathcal{NP}$难解性。为求解该路由问题,基于局部搜索及禁忌搜索,分别提出了两种路由算法Local-MPAR及Tabu-MPAR。此外,本章证明了Tabu-MPAR过程可以使得持有消息的节点集合最终达到预测投递概率最优的集合。仿真实验表明,两种MPAR算法,在机会网络中的综合移动模型WDM上优于DF算法及SimBet算法。

第二部分为基于消息传输收益的最优队列调度算法,其出发点在于,由于节点间带宽和接触时间受限所导致的连接有限的吞吐量会造成消息的传输失败,从而浪费了节点间宝贵的接触机会。通过定义概念“消息传输效用”,在PRoPHET的基础上加上数据项选择机制,得到改进后的路由算法Throughput。仿真结果表明,该数据项选择机制能够在消息投递率,消息投递时延,网络开销三方面较大程度的改进路由性能表现。

第三部分为基于跳数的启发式距离向量算法。提出了一种基于跳数的启发式路由协议HCH,利用启发函数,基于跳数信息进行消息投递跳数预测。利用滑动窗口机制,可以动态更新记录平均跳数的矩阵。基于此,定义了启发式函数用于预测当前节点到目的节点之间的潜在跳数。为了便于计算,定义了一种矩阵运算符,从而将跳数的估计过程转化为矩阵运算。仿真实验表明,本章提出的HCH算法在综合性能表现上优于Epidemic, S \& W以及PRoPHET。

