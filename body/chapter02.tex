\chapter{论文工作总览}

论文主要研究工作集中于\ref{chap:基于移动模式最优节点群组选取的路由算法}、\ref{chap:基于消息传输收益的最优队列调度算法}、\ref{chap:基于跳数的启发式距离向量算法}三章。

\begin{itemize}
\item 第\ref{chap:基于移动模式最优节点群组选取的路由算法}章提出了一个用于社交容迟网络的机会路由算法(Movement Pattern-aware Routing, MPAR)。MPAR算法从节点的移动记录中提取出节点(群组)的移动模式,并利用节点共同常访地点预测消息的投递概率。MPAR算法选取具有最优消息投递概率的节点群组,证明了该问题的NP-hard难解性,同时指出了求解该问题采用online算法的难度,并提出了启发式算法进行求解。主要技术方案如下:
\begin{enumerate}
\item \textbf{维护节点移动记录矩阵,提取节点移动模式,求预测投递率}。利用节点手机的移动记录信息,从中提取节点(群组)移动模式,并以此为依据选出最优中继节点群组。利用移动模式对应的共同常访地点集合,可求出消息的预测投递率。

\item \textbf{证明最优节点群组的求解为$\mathcal{NP}$难类问题}。
将子集和问题归约为节点群组选取问题的一个特例,从而证明了该问题的$\mathcal{NP}$难解性。

\item \textbf{基于tabu-search, 提出了启发式算法求解最优节点群组}。该章提出了两种路由算法:基于局部搜索的Local-MPAR以及基于禁忌搜索的Tabu-MPAR。其中Tabu-MPAR较Local-MPAR能够较为有效的跳出局部最优陷阱。
\end{enumerate}


\item 第\ref{chap:基于消息传输收益的最优队列调度算法}章设计了一种考虑带宽和链接持续时间的数据选择机制入手,尝试提高路由算法的性能。为了衡量每条消息的传输收益,定义了收益效用函数,并根据计算出的每条消息的传输收益值,将该问题建模为组合最优化问题,并采用动态规划(Dynamic Programming,DP)算法解决。主要技术方案如下:
\begin{enumerate}
\item \textbf{从消息投递率入手,为消息的每次传递定义收益值}。对于任意一对节点$a$和$b$, 分别计算出两节点共同持有消息副本时的投递概率,以及只有单一节点持有该消息时的投递概率,求差值以计算出本次传输的收益,为数据项在缓存中的调度提供参考依据。
\item \textbf{综合考虑带宽和节点间的接触持续时间,将数据选择问题建模为组合最优化问题}。结合第一点中所计算出的传输收益,以及节点的传输带宽和节点间的接触持续时间,把数据选择问题建模为组合最优化问题,并采用动态规划(Dynamic Programming,DP)算法解决.
\end{enumerate}

\item 第\ref{chap:基于跳数的启发式距离向量算法}章提出了一种基于跳数的启发式路由算法,并利用网络中的数据包携带节点之间平均跳数信息。定义了一个启发式函数用以估计消息距离目的节点所需的跳数。为了便于计算,将启发式函数的计算转化为矩阵乘法运算。
主要技术方案如下:
\begin{enumerate}
\item \textbf{尝试利用数据包来携带节点之间的跳数信息}。具体而言,在数据包头部写入该数据包所经过的节点,并记录节点与节点之间跳数。由此,当任意一条消息到达某节点时,利用\textbf{滑动窗口(slide-window)}技术更新其所记录的距离其它节点的平均跳数信息。

\item \textbf{定义启发函数用以估计消息从某节点到达其目的节点所需要的跳数}。函数如$
\label{eq:Hh}
\mathcal{H}(i,k) = hop(k) + h(i, d)
$其中$\mathcal{H}(i,k)$代表消息$k$若经过节点$i$到达目的节点$d$所需要的总跳数,由两部分组成:$hop(k)$代表消息$k$在到达节点$i$之前所经过的跳数(实际跳数),$h(i,d)$表示从当前节点$i$到目的节点$d$估计所需要经过的跳数(估计值)。

\item \textbf{将启发函数部分(即第二点中的$h(i,d)$部分)的计算转化为矩阵乘法问题。}将网络中节点建模为图论中的点,节点间的平均跳数信息建模为图论中的边权值,且为无向带权图,则该带权图表示为矩阵之后,可以利用矩阵相乘的方法求出$h(i,d)$的值,使运算更加简便。
\end{enumerate}
\end{itemize}