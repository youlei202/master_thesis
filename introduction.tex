\begin{introduction}

容迟网络或者容断网络(Delay/Disruption Tolerant Networks, DTN)是近年来无线网络领域内的一个研究热点,泛指部署在极端环境下由于节点的移动或者能量调度等原因而导致节点间只能间歇性进行通信甚至长时间处于中断状态的一类网络\upcite{Spyropoulos:2010tc,王:2013wq,杨:2012vw}。其概念起源于星际网络 (Interplanetary Internet/IPN),主要目的是为了将因特网中的协议应用到星际之间的网络互联中\upcite{肖:2007um}。为此国际互联网研究任 务组(Internet Research Task Force, IRTF)专门成立了星际网络研究小组 (Interplanetary Internet Research Group/IPNRG)。 星际网络中消息传输的主 要问题是长传播时延,其过长的传播时延会导致现有的网络协议失效\upcite{Maitreyi2013,Miao2013,Martin-Campillo2013,Desta:2013gj,Wei:2013da,Shrestha:2013fj, Aijaz:2013gq}。非对称的 带宽以及低比特率等等也都给网络协议设计带来了非常大的困难和挑战\upcite{杨:2009tc,袁:2009vf}。

    在 2001 和 2002 年,IPN 的研究者初次尝试将 IPN 的架构应用于其它一些陆地 上的挑战性网络中,如用于发展中国家偏远地区通信和 Internet 接入服务的信息网 络、湖泊环境下的水声传感器网、野生动物追踪网络、以及高速行驶的车辆组成的 车辆 Ad Hoc 网络等\upcite{Spyropoulos2007,Brunner2005,Wang2007,Xu:2009fx}。这些网络具有间歇连接、时延极高、频繁割裂、非对称数据率、 安全性差、较高的误码率与丢包率以及异构互联等特性\upcite{Mahendran2013,Zhu2013a,Li2013,Talipov2013,Borrego2013,Gong:2013cy,Jiang:2013df}。然而与 IPN 不同的是,陆 地上的挑战性网络由于节点的强移动性等特点,其更加强调节点之间连接的频繁中断(Disruption),而 IPN 由于节点之间具有极远的距离,其更加强调节点之间传播的高时延(Delay)。在 2004 年,Kevin Fall 在 SIGCOMM 上提出了一种用于此类挑战性网络的一种架构方式,自此容迟网络引起了空前的广泛关注。随着相关研究成果不断涌现,Delay/Disruption Tolerant Network 这一名称逐渐被研究者们接受。此外,国际互联网研究任务组(Internet Research Task Force, IRTF)还专门成立了容迟网络研究小组(Delay-Tolerant Networking Research Group, DTRNG。研究者在DTN领域开展了广泛的研究工作与试验部署,其应用涵盖了传统Internet之外的许多通讯网络,例如英国的UK-DMC(United Kingdom Disaster Monitoring Constelation),瑞典的SNC(Sami Network Connectivity),以及在非洲实施的斑马监测网络,外太空的星际网络(Inter-planetary networks)、用于发展中国家偏远地区通信和Internet接入服务的信息网络、湖泊环境下的水声传感器网、以及高速行驶的车辆组成的车辆Ad Hoc网络等。目前,容延网络已经成为网络界最为热门的研究课题之一\upcite{Ferretti2013,Ginzboorg2013,Cha2013,Mtibaa2013a,Lin2013,Wei:2013dr,Wong:2013jx,Li:2013fl,Xiao:2013bv,Sundararaj}。

    为了实现这些异构挑战性网络间的互操作以及消息的可靠传输,DTN引入了捆绑层(bundle layer)\upcite{Ronan2011}并采用了保管传递(custody transfer)和存储-携带-转发(store-carry-forward)机制\upcite{Voyiatzis2012,Kumar2012,Parikh2005,Ott2005,Boloni:2011tj}。DTN作为未来互联网络发展的一个新方向,在环境监测、交通管理、水下探测和发展中国家偏远地区网络基础建设具有广泛的应用前景和实用价值。DTN主干网(DTNBone)是DTN的实现之一\upcite{Zhang2011},正如已经投入使用的多播主干网(MBone)一样,DTNBone的主要目的是建立一个世界范围的运行DTN协议的主机集合,又叫节点集,研究者可以依此开展有关DTN的各种研究\upcite{Flocchini2013,Baier:2012hb,Moreira:2012fa,Li2011,}。目前搭建DTNBone 使用的主要协议集主要有DTN2, ION, LTPlib, Spindle3, IBR-DTN等。为了从应用层的角度来研究DTN,研究者们在现有的DTN协议栈的基础上实现了几项有关应用,如Ohio University Simple Bundle Protocol API,DTN Tic Tac Toe Application,DTN新闻应用以及DTN摄像机。

如何做出正确高效的路由选择一直是无线网络领域内的关键技术和主要研究课题,然而传统的基于TCP/IP的Internet路由协议、移动Ad Hoc网络和无线传感网络的路由协议均很难在容迟网络中工作\upcite{Zhu2013,Conti:2014gg,Mehmeti:2014gt,Zhang:2014dj,Zhu:2014il,Wang:2014tt}。其原因是TCP/IP以及无线传感网络中的路由协议,其假设某一时刻存在一个稳定连通的端到端的通信链路, 但是这一假设在容迟网络中不再成立\upcite{Medjiah:2014hr,Xia:2013eb,Abdelkader:2013jl,WjHsu:2013vb,Jedari:2013uo}。由于容迟网络中的节点具有高度移动性并且稀疏部署,这将导致容迟网络中的拓扑结构动态变化以及链路时断时续,因此在某个时刻或者某段时间内不存在端到端的路径。正是由于传输链路的这种时变性和不确定性使得容迟网络中的路由研究是一项挑战性的课题,因此设计可靠有效的容迟网络路由算法来提高网络连接性、降低能量消耗和时延、增加消息投递率就成为容迟网络研究的一个核心问题\upcite{Rothfus:2013gj,Deng:2013fr,Wei:2013hc,Fan:2013fv}。容迟网络中的路由算法研究作为一项关键技术成为新一代无线通信网络研究领域备受关注的前沿热点课题\upcite{XinKang:2013ui,XiaoChen:2013to,Wu:2013tk,PUTournoux:2013th,Ciobanu:2013gb,Yuan:2013hj}。

    目前DTN的路由方面的研究主要可以分为单播路由及多播路由两类\upcite{Hu:2013tw,Zhu2013a,Anthonysamy:2013wf},迄今大部分的研究成果主要集中在单播路由上,多播路由近几年逐渐被研究者所重视。单播路由可以分为复制策略(naive replication strategy),转发策略(utility based forwarding strategy), 混合策略(hybrid strategy), 基础设施辅助策略(infrastructure assistance strategy)以及编码策略(coding based strategy)五种;多播路由主要可以分为基于域内(intra-domain)的多播路由和基于域间(inter-domain)的多播路由。然而通过对容迟网络路由算法相关的文献,特别是近几年来的主要研究成果进行了总结发现,目前部分路由算法中所采用的节点移动模型过于理想化,节点的移动模式单一,缺乏实用性;且在一些特殊环境中,能量及缓存是更大的限制,应当为此单独设计算法优化;部分路由协议缺乏多项评估指标的综合考虑,往往在个别指标上性能优越,但无法优化多项指标,网络整体性能难以获得极大的提升。 

    国内目前仍然处于跟踪研究的初始阶段,急需在这一领域展开必要的研究并取得实质性的成果。鉴于以上总结,目前亟需设计基于多策略的容迟网络路由算法研究,以克服目前容迟网络路由算法研究中所存在的局限性。 本课题拟从上文所述的局限性出发,结合具体的应用场景,设计并实现对网络环境变化自适应的路由算法。




\end{introduction}
