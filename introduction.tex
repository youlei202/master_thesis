\begin{introduction}

首段先从无线网络分类讲起,然后撇开较成熟的网络,而转到移动无线网上,此外再介绍4G,5G技术及亟待攻克的问题,然后,介绍一下引言的结构安排。


\section*{I. 无线网络分类概述}
\addcontentsline{toc}{section}{I. 无线网络分类概述}
主要参考来源:Wikipedia, 介绍无线网络分类,及其区别

\section*{II. 移动无线网研究历程——从MANET到PCN}
\addcontentsline{toc}{section}{II. 移动无线网研究历程——从MANET到PCN}
主要参考来源:Mobile Ad Hoc Networking \cite{Conti:2014gg}

\section*{III. 相辅相成——机会网与蜂窝网技术融合}
\addcontentsline{toc}{section}{III. 相辅相成——机会网与蜂窝网技术融合}
其中一项,可以介绍一下opportunistic offloading

\section*{IV. 核心技术简介:机会路由,数据分发与能耗优化}
\addcontentsline{toc}{section}{IV. 核心技术简介:机会路由,数据分发与能耗优化}

\subsection*{IV-A. 机会路由}
\addcontentsline{toc}{subsection}{IV-A. 机会路由}

\subsection*{IV-B. 数据分发}
\addcontentsline{toc}{subsection}{IV-B. 数据分发}

\subsection*{IV-C. 能耗优化}
\addcontentsline{toc}{subsection}{IV-C. 能耗优化}


\end{introduction}
