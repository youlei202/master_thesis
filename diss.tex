%%==================================================
%% diss.tex for SJTU Master Thesis
%% based on CASthesis
%% modified by wei.jianwen@gmail.com
%% version: 0.3a
%% Encoding: UTF-8
%% last update: Dec 5th, 2010
%%==================================================

% 字号选项: c5size 五号(默认) cs4size 小四
% 双面打印(注意字号设置)
\documentclass[cs4size, a4paper, twoside]{sjtuthesis} 
% 单面打印(注意字号设置)
%\documentclass[cs4size, a4paper, oneside, openany]{sjtuthesis} 

% \usepackage[sectionbib]{chapterbib}%每章都用参考文献

\usepackage{algorithm}
\usepackage{graphicx}
\usepackage{array}
\usepackage{amsmath}
\usepackage{amssymb}
\usepackage{amsthm}
\usepackage{mathrsfs}
\usepackage{bm}
\usepackage{bbm}
\usepackage{bbold}
\usepackage{xfrac}
\usepackage{algorithmic}
\usepackage{upgreek}
\usepackage{multirow}
\usepackage{arydshln}
\usepackage{subfigure}

\newboolean{DOIT}
\setboolean{DOIT}{false}%编译某些只想自己看的内容,编译true,否则false

%% 行距缩放因子(x倍字号)
\renewcommand{\baselinestretch}{1.3}

% 设置图形文件的搜索路径
\graphicspath{{figure/}{figures/}{logo/}{logos/}{graph/}{graphs}}

%%========================================
%% 在sjtuthesis.cls中定义的有用命令
%%========================================
% \cndash 中文破折号
% 数学常量
% \me 对数常数e
% \mi 虚数单位i
% \mj 虚数单位j
% \dif 直立的微分算符d为直立体。
% 可伸长的数学箭头、等号
% \myRightarrow{}{}
% \myLeftarrow{}{}
% \myBioarrow{}{}
% \myLongEqual{}{}
% 参考文献
% \upcite{} 上标引用
%%========================================


\begin{document}

\renewcommand{\textfraction}{0.15} 
\renewcommand{\topfraction}{0.85} 
\renewcommand{\bottomfraction}{0.65} 
\renewcommand{\floatpagefraction}{0.60} 

%%%%%%%%%%%%%%%%%%%%%%%%%%%%%% 
%% 封面
%%%%%%%%%%%%%%%%%%%%%%%%%%%%%% 

% 中文封面内容(关注内容而不是形式)
\title{移动容迟网络中自适应路由算法的设计与实现}
\author{由\quad{}磊}
\advisor{魏长江教授,李建波副教授}
\degree{硕士}
\defenddate{2015年6月16日}
\school{青岛大学}
\institute{信息工程学院}
\studentnumber{2012020406}
\major{软件工程}

% 英文封面内容(关注内容而不是表现形式)
\englishtitle{\XeTeX/\LaTeX\, Template for QDU Master Degree Thesis \version}
\englishauthor{\textsc{Lei You}}
\englishadvisor{Prof. \textsc{Changjiang Wei}}
\englishschool{Qingdao University}
\englishinstitute{\textsc{Information Engineering College} \\
  \textsc{Qingdao University} \\
  \textsc{Shandong, P.R.China}}
\englishdegree{Master}
\englishmajor{Software Engineering}
\englishdate{Jun. 16th, 2015}

% 封面
\maketitle

% 英文封面
%\makeenglishtitle



%%%%%%%%%%%%%%%%%%%%%%%%%%%%%% 
%% 前言
%%%%%%%%%%%%%%%%%%%%%%%%%%%%%% 
\frontmatter

% 摘要
%%==================================================
%% abstract.tex for SJTU Master Thesis
%% based on CASthesis
%% modified by wei.jianwen@gmail.com
%% version: 0.3a
%% Encoding: UTF-8
%% last update: Dec 5th, 2010
%%==================================================

\begin{abstract}


\begin{figure}[!b]
  \keywords{\large \textbf{机会网络\quad 蜂窝网络 \quad 路由协议 \quad 数据分发 \quad 优化算法}}
  \end{figure}
\end{abstract}



\begin{englishabstract}



\begin{figure}[!b]
\englishkeywords{\large \textbf{opportunistic network; cellular network; routing protocol; data dissemination; optimization algorithm}}
\end{figure}
\end{englishabstract}


% 目录
\tableofcontents

\theoremstyle{plain}
\newtheorem{definition}{定义}
\newtheorem{theorem}{定理}
\newtheorem{lemma}{引理}
\newtheorem{proposition}{命题}
\newtheorem{postulation}{公设}

%%%%%%%%%%%%%%%%%%%%%%%%%%%%%% 
%% 正文
%%%%%%%%%%%%%%%%%%%%%%%%%%%%%% 
\mainmatter


%%引言
\begin{introduction}

容迟网络或者容断网络(Delay/Disruption Tolerant Networks, DTN)是近年来无线网络领域内的一个研究热点,泛指部署在极端环境下由于节点的移动或者能量调度等原因而导致节点间只能间歇性进行通信甚至长时间处于中断状态的一类网络\upcite{Spyropoulos:2010tc,王:2013wq,杨:2012vw}。其概念起源于星际网络 (Interplanetary Internet/IPN),主要目的是为了将因特网中的协议应用到星际之间的网络互联中\upcite{肖:2007um}。为此国际互联网研究任 务组(Internet Research Task Force, IRTF)专门成立了星际网络研究小组 (Interplanetary Internet Research Group/IPNRG)。 星际网络中消息传输的主 要问题是长传播时延,其过长的传播时延会导致现有的网络协议失效\upcite{Maitreyi2013,Miao2013,Martin-Campillo2013,Desta:2013gj,Wei:2013da,Shrestha:2013fj, Aijaz:2013gq}。非对称的 带宽以及低比特率等等也都给网络协议设计带来了非常大的困难和挑战\upcite{杨:2009tc,袁:2009vf}。

    在 2001 和 2002 年,IPN 的研究者初次尝试将 IPN 的架构应用于其它一些陆地 上的挑战性网络中,如用于发展中国家偏远地区通信和 Internet 接入服务的信息网 络、湖泊环境下的水声传感器网、野生动物追踪网络、以及高速行驶的车辆组成的 车辆 Ad Hoc 网络等\upcite{Spyropoulos2007,Brunner2005,Wang2007,Xu:2009fx}。这些网络具有间歇连接、时延极高、频繁割裂、非对称数据率、 安全性差、较高的误码率与丢包率以及异构互联等特性\upcite{Mahendran2013,Zhu2013a,Li2013,Talipov2013,Borrego2013,Gong:2013cy,Jiang:2013df}。然而与 IPN 不同的是,陆 地上的挑战性网络由于节点的强移动性等特点,其更加强调节点之间连接的频繁中断(Disruption),而 IPN 由于节点之间具有极远的距离,其更加强调节点之间传播的高时延(Delay)。在 2004 年,Kevin Fall 在 SIGCOMM 上提出了一种用于此类挑战性网络的一种架构方式,自此容迟网络引起了空前的广泛关注。随着相关研究成果不断涌现,Delay/Disruption Tolerant Network 这一名称逐渐被研究者们接受。此外,国际互联网研究任务组(Internet Research Task Force, IRTF)还专门成立了容迟网络研究小组(Delay-Tolerant Networking Research Group, DTRNG。研究者在DTN领域开展了广泛的研究工作与试验部署,其应用涵盖了传统Internet之外的许多通讯网络,例如英国的UK-DMC(United Kingdom Disaster Monitoring Constelation),瑞典的SNC(Sami Network Connectivity),以及在非洲实施的斑马监测网络,外太空的星际网络(Inter-planetary networks)、用于发展中国家偏远地区通信和Internet接入服务的信息网络、湖泊环境下的水声传感器网、以及高速行驶的车辆组成的车辆Ad Hoc网络等。目前,容延网络已经成为网络界最为热门的研究课题之一\upcite{Ferretti2013,Ginzboorg2013,Cha2013,Mtibaa2013a,Lin2013,Wei:2013dr,Wong:2013jx,Li:2013fl,Xiao:2013bv,Sundararaj}。

    为了实现这些异构挑战性网络间的互操作以及消息的可靠传输,DTN引入了捆绑层(bundle layer)\upcite{Ronan2011}并采用了保管传递(custody transfer)和存储-携带-转发(store-carry-forward)机制\upcite{Voyiatzis2012,Kumar2012,Parikh2005,Ott2005,Boloni:2011tj}。DTN作为未来互联网络发展的一个新方向,在环境监测、交通管理、水下探测和发展中国家偏远地区网络基础建设具有广泛的应用前景和实用价值。DTN主干网(DTNBone)是DTN的实现之一\upcite{Zhang2011},正如已经投入使用的多播主干网(MBone)一样,DTNBone的主要目的是建立一个世界范围的运行DTN协议的主机集合,又叫节点集,研究者可以依此开展有关DTN的各种研究\upcite{Flocchini2013,Baier:2012hb,Moreira:2012fa,Li2011,}。目前搭建DTNBone 使用的主要协议集主要有DTN2, ION, LTPlib, Spindle3, IBR-DTN等。为了从应用层的角度来研究DTN,研究者们在现有的DTN协议栈的基础上实现了几项有关应用,如Ohio University Simple Bundle Protocol API,DTN Tic Tac Toe Application,DTN新闻应用以及DTN摄像机。

如何做出正确高效的路由选择一直是无线网络领域内的关键技术和主要研究课题,然而传统的基于TCP/IP的Internet路由协议、移动Ad Hoc网络和无线传感网络的路由协议均很难在容迟网络中工作\upcite{Zhu2013,Conti:2014gg,Mehmeti:2014gt,Zhang:2014dj,Zhu:2014il,Wang:2014tt}。其原因是TCP/IP以及无线传感网络中的路由协议,其假设某一时刻存在一个稳定连通的端到端的通信链路, 但是这一假设在容迟网络中不再成立\upcite{Medjiah:2014hr,Xia:2013eb,Abdelkader:2013jl,WjHsu:2013vb,Jedari:2013uo}。由于容迟网络中的节点具有高度移动性并且稀疏部署,这将导致容迟网络中的拓扑结构动态变化以及链路时断时续,因此在某个时刻或者某段时间内不存在端到端的路径。正是由于传输链路的这种时变性和不确定性使得容迟网络中的路由研究是一项挑战性的课题,因此设计可靠有效的容迟网络路由算法来提高网络连接性、降低能量消耗和时延、增加消息投递率就成为容迟网络研究的一个核心问题\upcite{Rothfus:2013gj,Deng:2013fr,Wei:2013hc,Fan:2013fv}。容迟网络中的路由算法研究作为一项关键技术成为新一代无线通信网络研究领域备受关注的前沿热点课题\upcite{XinKang:2013ui,XiaoChen:2013to,Wu:2013tk,PUTournoux:2013th,Ciobanu:2013gb,Yuan:2013hj}。

    目前DTN的路由方面的研究主要可以分为单播路由及多播路由两类\upcite{Hu:2013tw,Zhu2013a,Anthonysamy:2013wf},迄今大部分的研究成果主要集中在单播路由上,多播路由近几年逐渐被研究者所重视。单播路由可以分为复制策略(naive replication strategy),转发策略(utility based forwarding strategy), 混合策略(hybrid strategy), 基础设施辅助策略(infrastructure assistance strategy)以及编码策略(coding based strategy)五种;多播路由主要可以分为基于域内(intra-domain)的多播路由和基于域间(inter-domain)的多播路由。然而通过对容迟网络路由算法相关的文献,特别是近几年来的主要研究成果进行了总结发现,目前部分路由算法中所采用的节点移动模型过于理想化,节点的移动模式单一,缺乏实用性;且在一些特殊环境中,能量及缓存是更大的限制,应当为此单独设计算法优化;部分路由协议缺乏多项评估指标的综合考虑,往往在个别指标上性能优越,但无法优化多项指标,网络整体性能难以获得极大的提升。 

    国内目前仍然处于跟踪研究的初始阶段,急需在这一领域展开必要的研究并取得实质性的成果。鉴于以上总结,目前亟需设计基于多策略的容迟网络路由算法研究,以克服目前容迟网络路由算法研究中所存在的局限性。 本课题拟从上文所述的局限性出发,结合具体的应用场景,设计并实现对网络环境变化自适应的路由算法。




\end{introduction}

%% 各章正文内容
\chapter{研究现状}

DTN为了实现这些异构挑战性网络间的互操作以及消息的可靠传输,引入了捆绑层(bundle layer)并采用了保管传递(custody transfer)和存储-携带-转发(store-carry-forward)机制。DTN作为未来互联网络发展的一个新方向,在环境监测、交通管理、水下探测和发展中国家偏远地区网络基础建设具有广泛的应用前景和实用价值。DTN主干网(DTNBone)是DTN的实现之一,正如已经投入使用的多播主干网(MBone)一样,DTNBone的主要目的是建立一个世界范围的运行DTN协议的主机集合,又叫节点集,研究者可以依此开展有关DTN的各种研究。目前搭建DTNBone 使用的主要协议集有DTN2, ION, LTPlib, Spindle3, IBR-DTN等。为了从应用层的角度来研究DTN,研究者们在现有的DTN协议栈的基础上实现了几项有关应用,如Ohio University Simple Bundle Protocol API,DTN Tic Tac Toe Application,DTN新闻应用以及 DTN摄像机。

DTN的路由方面的研究越来越受到国内外学者的重视,主要可以分为单播路由及多播路由两类,迄今大部分的研究成果主要集中在单播路由上,多播路由近几年逐渐被研究者所重视。单播路由可以分为复制策略(naive replication strategy),转发策略(utility based forwarding strategy), 混合策略(hybrid strategy), 基础设施辅助策略( infrastructure assistance strategy)以及编码策略(coding based strategy)五种;多播路由主要可以分为基于域内(intra-domain)的多播路由和基于域间(inter-domain)的多播路由。

\section{单播路由}
\subsection{复制策略}
复制策略的特点是节点在向中继节点进行消息复制时,不考虑中继节点的选择问题。常见的方法主要有洪泛(flooding)以及受控洪泛(controlled flooding)。著名的传染病(Epidemic)路由算法\cite{Vahdat2000}是基于洪泛的一个代表性算法,该算法模仿生物环境中传染性病毒的传播方式,体现在DTN中,每个节点维护一个消息总结向量,当两个节点能够连接时通过交换消息向量来彼此交换缺少的消息。然而过度冗余的消息副本会带来网络拥塞,从而增大时延和丢包率。文献\cite{Grossglauser2002}提出的Direct Delivery(DD)算法则可被视为是基于洪泛的另一个极端情况,该算法同样不考虑中继节点的选择——不向任何中继节点复制消息,每个节点只将消息直接传递给目的节点,故具有很低的投递率,只作为研究路由算法的一种参考。文献\cite{Grossglauser2002}还提出了一种两跳中继多次复制(Two-Hop-Relay)的路由策略,源节点将消息复制给与其关联的T个邻居节点,这些邻居节点不再进行复制操作,而是将这些消息直接传输至与之有关联的目的节点。文献\cite{Spyropoulos2005}提出了Spray and Wait算法,源节点持有T份拷贝,并向遇到的节点分发所持有的拷贝,若节点持有的拷贝数量大于1,则继续进行分发,当节点只持有消息的一个拷贝时,采用DD算法将消息传输到目的节点。文献\cite{Spyropoulos2007a}给出了一种特殊的“Spray”方法,每次分发一半的拷贝给第一个遇到的节点,使整个Spray阶段呈二叉树状,当节点移动独立同分布时,其期望时延最优。

\subsection{转发策略}

单纯使用转发策略的路由,往往也称为单副本路由,是根据网络的拓扑知识来指定一个或几个用于衡量节点作为中继转发某消息的合适程度或链路的传输时延等的效用指标,制定效用指标的目的,往往是为了在选路过程中优化某一个或者某几个目标。单副本转发策略必须确保据此所选出的中继节点能够比源节点更好的胜任消息投递工作,如果可以设定出能够准确描述节点属性的效用指标,则可以保证在网络中只有一个消息的副本在传输时达到一个较为理想的投递率,保证可靠传输。目前除了星际网络之外,很少有能够预测节点移动以及网络拓扑变化,这给指定精确效用指标带来了困难。文献\cite{Jain2004}提出的适用于星际网络的路由算法是该策略的典型代表,它将容迟网络建模为有向多重图,将网络的动态拓扑信息抽象为知识ORACLE集。以知识ORACLE作为路由问题的输入,研究路由算法性能与ORACLE间的关系。ORACLE分为接触总结、接触、排队和流量需求四种类型,根据所需ORACLE的不同,文献\cite{Jain2004}提出了FC、MED、ED、EDLQ、EDAQ和LP共六个算法。其中FC算法最简单,基于零知识输入,以时间作为效用指标,每个节点选择与自己接触最早的节点进行转发。其它五个算法的效用指标都是基于链路状态时延而指定的,根据能够利用的ORACLE知识的多少,效用指标的准确程度也不同,MED算法基于网络中的节点接触总结,采用Dijkstra算法求解最优路由,由于节点接触总结信息是固定的,故该算法不是一个自适应路由算法,从这种意义上讲,若网络情况稍有变化,则效用指标就不再能准确反映网络情况。ED算法则根据网络中所有节点的接触信息采用改进的Dijkstra算法求解最优路由。而EDLQ和EDAQ路由算法则是在ED算法的基础上分别引入局部节点缓存队列和全局缓存队列信息,进一步优化性能。而LP路由算法则进一步的引入了流量需求,将最优路径转化为一个线性规划问题进行求解,然而这种方法可行性不足,主要作为研究参考。文献\cite{Guo2013}提出的PASR(Prediction Assisted Single-copy Routing)收集并维护在较短时间内获取的网络连接信息,然后使用一个离线的贪心算法求解出底层网络移动性的有关特点,描述出最优路径所具有的特点并依此给出一个有效利用历史信息的导引方法,PASR根据此方法,利用一个在线算法来计算路由。为了取得较好的网络性能通常需要全局或者部分总结的拓扑信息,而这些信息在大部分强调节点随机移动性容迟网络中往往不可能预先获知,这就要求网络中节点按需(on-demand)获取网络拓扑,,目前常用的方法是依靠一个链路状态协议(link state protocol)进行拓扑知识分发。文献\cite{Jones2007}提出的MEED算法即是如此,其进行路由所需要的全局链路拓扑信息单独由一个链路状态协议维持。MEED在MED算法之上做了改进,MED算法假设网络中的拓扑知识集能够获得,然而在实际的情况中可行性不大。MEED基于滑动窗口对链路的期望时延进行评估,文献\cite{Jones2007}还提出了一种与源路由和按需路由不同的一种路由策略,即逐跳路由(per-hop routing),利用该策略能够更好的利用最近所出现的可用链路。

\subsection{混合策略}

与基于复制策略的路由算法以及只利用单拷贝转发策略的路由算法相比,混合策略的路由算法更为灵活,既利用了复制策略多副本的并行性,又优化选择了较为合适的中继节点。文献\cite{Spyropoulos2007a}提出的Spray \& Focus算法是一种典型的混合策略路由算法,在Focus阶段,算法实行聚焦搜索,有选择的进行副本的转发投递。文献\cite{Erramilli2008}提出的授权转发(delegation forwarding/DF),将DTN中路由问题转化为停止问题(stopping problem)进行研究,其基本策略是对节点定义效用函数,用以描述节点对于转发某消息的合适程度,每当进行一次复制操作后,两个节点都将更新自己的效用函数,从而使今后选择转发节点的指标越来越严格,有效的控制了副本的数量。该文献证明了,采用授权转发协议产生的副本数量可由$O(n)$降低到$O(\sqrt{n})$。文献\cite{Liu:2009uy}提出的OOF(Optimal Opportunistic Forwarding)算法结合了跳数和副本存活时间来定义了副本的投递概率,并利用最优停止理论递推的对投递概率进行计算,获得期望上的最优投递率。由于OOF算法中,其需要维护的投递概率表是四维的,计算及更新起来具有较大的开销,为了解决此问题,文献\cite{Liu:2009uy}还提出了OOF—算法,即不再考虑副本存活时间,从而使概率表降低一维,模拟实验表明其表现只略微弱于OOF算法。文献\cite{Burns:2005gi}提出的MV(Meetings and Visits)则利用节点间的相遇概率来描述消息传输的成功概率,任意两个节点间的相遇概率作为这对节点的传输概率,在此基础上通过递归的方式计算多跳传输的成功概率。两个节点相遇时,交换各自的消息以及传输概率信息,通过比较,节点仅向传输概率更高的中继节点复制消息。文献\cite{El-Azouzi2013}将进化策略(Evolutionary Games)用于非合作性的受控转发(controlled-forwarding)中,在这种方法中,节点可以选择不同的策略来作为中继节点参与转发,引入效用函数来权衡期望收益和能量消耗,文献\cite{El-Azouzi2013}证明了该方法可以以一种非集中式的方式进行,故适用于容迟网络选路。


\subsection{基础设施辅助策略}
基础设施辅助策略路由算法采用特殊的移动节点来负责容迟网络中节点间的消息寻路和传输,这些特殊节点通常称之为消息渡船(message ferry)和数据骡子(data mule)。这些节点通过控制自己的移动行为,按照一定的移动路线在网络相互割裂的区域进行移动,以“存储-携带-转发”(store-carry-forward)的方式辅助进行网络的消息转发,提高消息投递率。文献\cite{Zhao2004}提出的MF(Message Ferry)算法是基于拓扑知识策略的典型代表,该算法通过引入一个称为消息渡船的特殊移动节点并通过规划该节点的移动路线,协助区域间的消息传输。作为MF的扩展,文献\cite{Zhao2005}提出了四种方法,考虑了引入多个消息渡船的网络,以满足流量需求且传输延迟最小为优化目标,计算出消息渡船的最优移动路线,这四种方法分别为单路由算法(SIngle Route Algorithm/SIRA),多路由算法(MUlti Route Algorithm/MURA),节点中继算法(Node Relaying Algorithm/NRA)以及渡船中继算法(Ferry Relaying Algorithm/FRA)。文献\cite{Zhao2005}还从另一个角度审视了其所提出的MF算法, 所改进的方法是不引入任何受控辅助接点来引渡消息,其所需要的渡船节点是由网络中原有节点选举产生的。文献\cite{Zhang:2007bq}基于旅行商问题(Travel Salesman Problem/TSP)提出了另外一种渡船算法,其核心主要侧重于设计引渡节点路由,从而平衡消息传递率以及节点有限的缓存资源。文献\cite{Banerjee:2007hm}所利用的投掷盒(Throwbox)是一种低成本,电池驱动的短程小存储量设备,由于其易于部署,故很适合用作引渡节点。当两个不同的节点都经过投掷盒所覆盖的范围时,先到达的节点可以把消息缓存在其中,以便将该消息传输到后到达的节点,这实际上也是一种保管传输策略。辅助节点的部署问题也是一个难点,文献\cite{He:2010ks}提出的一种方法,其将传输时延和副本数目纳入考虑之中,结合效用函数利用贪心法选择位置来进行辅助节点部署,该算法具有二次多项式的时间复杂度。


\subsection{编码策略}
编码策略实质上是对DTN间歇连接特性予以补偿的一种机制,其并非是一种路由策略,但可以和其它常用路由策略混合以提高投递率。常见的编码策略有网络编码和擦除码,前者允许中继节点对消息进行编码,而后者只允许源节点对消息进行编码。文献\cite{Wang:2005ka}使用了擦除码和复制策略来进行路由,在简单的复制策略中,消息会从源节点传给r个中继节点,而使用纠删码后,会产生kr个同等大小的编码块,并将复制给kr个中继节点,与普通的复制路由策略相比,加入了擦出码策略的方法采用了更多的中继节点,只要这些中继节点中有多于k个节点具有更小的延迟,则其表现就超过只用普通的复制策略的路由,其本质即是在消息所需总带宽不变的情况下增加其之间的并行性。文献\cite{Jain2005}使用网络编码结合概率算法进行路由,当消息到达中继节点时,会被中继节点与其它接受到的消息混合进行线性编码产生一个新的消息,目标节点收到消息时再进行解码,从而减少消息传输的次数,提高消息分发概率。


\section{多播路由}
多播路由可以分为基于域内(intra-domain)的多播路由和基于域间(inter-domain)的多播路由。文献\cite{Zhao2005a}是域内多播协议的典型代表,首先定义了三种多播语义模型,进而提出了五种路由算法:分别是基于静态树多播路由(static tree-based routing,简称STBR),基于动态树的多播路由(dynamic tree-based routing,简称DTBR),基于转发组的路由(GBR)、基于广播的多播路由(BBR)和基于单播的多播路由(UBR),并研究这些路由算法随着组成员知识ORACLE的不同而产生的性能变化。DTBR算法通常需要全局的拓扑知识来构建一个动态多播树,并且树中每个中间节点的多播成员接收列表由其上游节点决定,这一特点决定了DTBR算法不能很好地利用现有新出现的消息投递机会。为了克服这一缺点,文献\cite{Ye2006}提出了OS-Multicast算法,该算法是一种基于动态树的按需式情形感知多播路由算法,动态树的每个中间节点维护一个所有多播成员的接受列表,以更好的利用最新出现的投递机会,但代价是网络中会有多余的多播消息副本在传输。文献\cite{Chuah2009}提出了CAMR(Context Aware Multicast Routing)算法,该算法是一种基于节点密度的自适应多播路由算法。该算法试图通过获得一些额外的网络知识例如节点的速度和位置以获取更好的网络性能,同时该算法综合利用了节点的高发射功率和消息渡船机制以取得更高的消息投递率。文献\cite{Wang:2012up}提出了一种基于动态多播树非复制策略的多播路由算法,该算法提出了比较-分割原则(compare and split)并同时利用节点的活动率水平(active rate level)和接触率水平(contact rate level)来指导动态多播树的构建。文献\cite{Gao:2009wf}第一次提出基于社会网络(social network)的相关技术来研究容迟网络中的多播问题,该文献利用社会网络中的两个关键概念:社区(Communities)和聚集性(Centrality)计算DTN多播过程中的中继节点选择问题,并最终将该问题统一为常见的背包问题(knapsack problem)。基于域间的多播路由算法通常在每个DTN区域选择一个称为域首(domain leader)的节点来负责此区域多播消息的汇总和分发,这样整个DTN区域可划分为两层:域首层和域层。多播过程分为三步:(1)源节点将消息通过域内多播协议发送至源区域所在的域首,(2)源域首经过域间多播协议将消息传送至多播接收成员所在的各个区域的域首 (3)各个接收节点域首再经过域内多播协议传送至多播目的节点。文献\cite{Ye:2007cf}所提出的SHIM(scalable hierarchical inter-domain multicast)算法和文献\cite{Yang:2008hn}所提出的FBIMR(ferry-based inter-domain multicast routing)算法是典型的域间多播路由算法。
除了上述所提的路由算法外,还有不少与路由直接或间接相关的研究成果,例如芬兰赫尔辛基理工大学开发的the ONE(Opportunistic NEtwork Simulator)模拟器\cite{Keranen2009},该模拟器用于对DTN中的路由算法性能进行仿真、模拟和验证。文献\cite{Talipov2013}以智能手机系统为基础,利用存储-携带-转发方法,设计了基于bundle协议的信息分享的方法,该方法有效的降低了CPU以及电池的负载。

通过对容迟网络路由算法相关的文献,特别是近几年来的主要研究成果进行了总结发现:(1)复制策略可以增大消息的并行性和传递可靠性,有效的受控洪泛能够大大改善挑战性网络中的路由表现。(2)目前部分路由算法中所采用的节点移动模型过于理想化,节点的移动模式单一,缺乏实用性。(3)目前路由协议主要致力于优化投递率及传输实验,然而在一些特殊环境中,能量及缓存是更大的限制,应当为此单独设计算法优化(4)目前的路由协议缺乏多项评估指标的综合考虑,往往在个别指标上性能优越,但无法优化多项指标,网络整体性能难以获得极大的提升。因此需要利用新的分析工具研究容迟网络路由,同时考虑多个设计目标进行优化,建立基于多目标优化的高效路由协议.例如,以节点能量消耗、时延、传输率为目标,进行多目标决策,设计出最优路由协议。(5)国内目前仍然处于跟踪研究的初始阶段,急需在这一领域展开必要的研究并取得实质性的成果。鉴于以上总结,目前亟需设计基于多策略的容迟网络路由算法研究,以克服目前容迟网络路由算法研究中所存在的局限性。 

\chapter{论文工作总览}

论文主要研究工作集中于\ref{chap:基于移动模式最优节点群组选取的路由算法}、\ref{chap:基于消息传输收益的最优队列调度算法}、\ref{chap:基于跳数的启发式距离向量算法}三章。

\begin{itemize}
\item 第\ref{chap:基于移动模式最优节点群组选取的路由算法}章提出了一个用于社交容迟网络的机会路由算法(Movement Pattern-aware Routing, MPAR)。MPAR算法从节点的移动记录中提取出节点(群组)的移动模式,并利用节点共同常访地点预测消息的投递概率。MPAR算法选取具有最优消息投递概率的节点群组,证明了该问题的NP-hard难解性,同时指出了求解该问题采用online算法的难度,并提出了启发式算法进行求解。主要技术方案如下:
\begin{enumerate}
\item \textbf{维护节点移动记录矩阵,提取节点移动模式,求预测投递率}。利用节点手机的移动记录信息,从中提取节点(群组)移动模式,并以此为依据选出最优中继节点群组。利用移动模式对应的共同常访地点集合,可求出消息的预测投递率。

\item \textbf{证明最优节点群组的求解为$\mathcal{NP}$难类问题}。
将子集和问题归约为节点群组选取问题的一个特例,从而证明了该问题的$\mathcal{NP}$难解性。

\item \textbf{基于tabu-search, 提出了启发式算法求解最优节点群组}。该章提出了两种路由算法:基于局部搜索的Local-MPAR以及基于禁忌搜索的Tabu-MPAR。其中Tabu-MPAR较Local-MPAR能够较为有效的跳出局部最优陷阱。
\end{enumerate}


\item 第\ref{chap:基于消息传输收益的最优队列调度算法}章设计了一种考虑带宽和链接持续时间的数据选择机制入手,尝试提高路由算法的性能。为了衡量每条消息的传输收益,定义了收益效用函数,并根据计算出的每条消息的传输收益值,将该问题建模为组合最优化问题,并采用动态规划(Dynamic Programming,DP)算法解决。主要技术方案如下:
\begin{enumerate}
\item \textbf{从消息投递率入手,为消息的每次传递定义收益值}。对于任意一对节点$a$和$b$, 分别计算出两节点共同持有消息副本时的投递概率,以及只有单一节点持有该消息时的投递概率,求差值以计算出本次传输的收益,为数据项在缓存中的调度提供参考依据。
\item \textbf{综合考虑带宽和节点间的接触持续时间,将数据选择问题建模为组合最优化问题}。结合第一点中所计算出的传输收益,以及节点的传输带宽和节点间的接触持续时间,把数据选择问题建模为组合最优化问题,并采用动态规划(Dynamic Programming,DP)算法解决.
\end{enumerate}

\item 第\ref{chap:基于跳数的启发式距离向量算法}章提出了一种基于跳数的启发式路由算法,并利用网络中的数据包携带节点之间平均跳数信息。定义了一个启发式函数用以估计消息距离目的节点所需的跳数。为了便于计算,将启发式函数的计算转化为矩阵乘法运算。
主要技术方案如下:
\begin{enumerate}
\item \textbf{尝试利用数据包来携带节点之间的跳数信息}。具体而言,在数据包头部写入该数据包所经过的节点,并记录节点与节点之间跳数。由此,当任意一条消息到达某节点时,利用\textbf{滑动窗口(slide-window)}技术更新其所记录的距离其它节点的平均跳数信息。

\item \textbf{定义启发函数用以估计消息从某节点到达其目的节点所需要的跳数}。函数如$
\label{eq:Hh}
\mathcal{H}(i,k) = hop(k) + h(i, d)
$其中$\mathcal{H}(i,k)$代表消息$k$若经过节点$i$到达目的节点$d$所需要的总跳数,由两部分组成:$hop(k)$代表消息$k$在到达节点$i$之前所经过的跳数(实际跳数),$h(i,d)$表示从当前节点$i$到目的节点$d$估计所需要经过的跳数(估计值)。

\item \textbf{将启发函数部分(即第二点中的$h(i,d)$部分)的计算转化为矩阵乘法问题。}将网络中节点建模为图论中的点,节点间的平均跳数信息建模为图论中的边权值,且为无向带权图,则该带权图表示为矩阵之后,可以利用矩阵相乘的方法求出$h(i,d)$的值,使运算更加简便。
\end{enumerate}
\end{itemize}
\chapter{基于移动模式最优节点群组选取的路由算法}

基于节点的移动模式,本章提出构造最优节点群组作为中继节点群的路由算法。在许多社会容迟网络(Social Delay Tolerant Networks, SDTN)中,具有共同兴趣的移动用户往往访问一些与其兴趣相关的地点。研究表明,50\%的移动用户会在某一个特定的接入点(access point, AP)上花费约74\%的时间\cite{Henderson:2004ul}。换言之,节点往往具有频繁访问某一或某一部分地点(简称为常访地点)的特点。这些常访地点可被看做``连接''这些节点的枢纽。可以通过在常访地点部署缓存设备,用以辅助消息传递,例如投掷盒(throw-box)\cite{Ibrahim:2009we}等设备。缓存设备具有普通移动节点不具备的优势。首先,由于部署的缓存设备位置固定在常访地点,且节点往往在常访地点停留一段时间,所以节点与缓存设备之间具有较为稳定的连接。不同与此,

\section{系统模型及基本定义}

\subsection{网络模型}

\subsection{基本定义}

\section{路由问题概览}

\subsection{移动模式定义}

\subsection{路由相关的两个关键属性}

\subsubsection{投递概率}

\subsubsection{期望时延}

\subsection{路由问题形式化定义}

\section{$N_{opt}$搜索问题分析}

\subsection{计算复杂性证明}

\subsection{局部陷阱}

\subsection{基于禁忌搜索的求解方法}

\section{移动模式相关的最优化路由}

\subsection{Local-MPAR:基于局部搜索的路由算法}

\subsection{Tabu-MPAR:基于禁忌搜索的路由算法}

\section{仿真实验}

\subsection{自变量:消息生存时间}

\subsection{自变量:节点缓存}

\section{结论}







\chapter{基于消息传输收益的最优队列调度算法}

本章尝试改进经典的基于概率预测的路由算法PRoPHET \cite{AndersLindgren2004}。定义了“传输收益”概念,基于此,提出了传输收益最优化,吞吐量相关的概率路由问题(Optimal Throughput-Aware Probabilistic Routing)。该问题被建模为最优决策问题,并在本章中采用动态规划方法解决。在该模型下,提出了一种数据项选择机制,其具有如下特点:
\begin{enumerate}
\item 每一对节点所对应的潜在连接,其平均数据吞吐量可能比缓存中的消息大小要小。
\item 每次传输的能量消耗不可忽略,换言之,消息传输不成功会带来无收益的能量损耗。
\end{enumerate}

例如,针对某条消息的一次传输操作而言,若当前连接所允许的总数据量比消息的大小小很多,则该次传输终止时,消息并未被成功传输。针对该问题的一个解决办法,即是总是传输不超过该连接平均吞吐量大小的消息,从而避免浪费传输机会。退一步而言,即使连接的吞吐量完全足够传输每一条消息,传输不同消息所带来的收益也并不相同(比如投递率收益,及投递时延收益等),然而传输的机会是有限的,故针对每一次传输机会而言,传输不同的消息对路由的整体性能表现具有较大的影响。从这个角度而言,在机会网络中,为了提高路由性能表现,应当设计有效的数据项选择机制。

本章组织如下:第\ref{chap4:系统模型}节中介绍了系统模型及路由模型;第\ref{chap4:问题形式化}节中形式化定义了关键问题;第\ref{chap4:基于数据项选择的改进路由}节中利用设计的数据项选择机制对PRoPHET进行改进;第\ref{chap4:仿真实验}节为仿真实验;第\ref{chap4:本章小结}节概括了本章内容。

\section{系统模型}
\label{chap4:系统模型}

\begin{table*}
\centering
  \caption{数学符号}
    \label{tab:chap4_notations}
  \begin{tabular}{p{0.23\linewidth}<{\centering}p{0.7\linewidth}<{\centering}}
  \hline
    \textbf{Notation} & \textbf{Meaning}  \\
    \hline
   $\mathcal{N}$ & 网络节点集合 \\
   $n_i$ &  第 $i$ 号节点  \\
   $m_i$ &  第 $i$ 号消息 \\
   $TTL(i)$ & 消息 $m_i$ 的生存时间\\
   $S(i)$   & 消息 $m_i$ 的大小\\
    $\zeta(i)$ & 消息 $m_i$ 的传输收益\\
    $B_{(a,b)}$ & 节点 $n_a$ 与 $n_b$ 之间的带宽\\
   $t_{s_{(a,b)}}/t_{e_{(a,b)}}$   &  节点 $n_a$ and $n_b$ 最近一次连接的开始/结束时间\\
   $\tau_{a,b}$ & 节点 $n_a$ 与 $n_b$ 预测接触时间\\
  $P_{(a,b)}$ & $n_a$'s 节点 $n_a$ 与 $n_b$ 的预测接触概率 \\
  $ETH_{(a,b)}$ & 节点 $n_a$ 与 $n_b$ 连接的预测吞吐量\\
$P^{\{a,b\}}$ & 节点 $n_a$ 与 $n_b$ 的共同投递率\\
   \hline
  \end{tabular}
\end{table*}

数学符号如\tablename~\ref{tab:chap4_notations}所示。本模型的时间线设为离散时间,时间轴被划分为多个小的时间槽,每个时间槽的长度定义为一个单位时间。整个节点集合用符号$\mathcal{N}=\{n_i|n_i\in\mathcal{N},1\leq i<|\mathcal{N}|\}$。对于任意一条消息$m_k$,其消息生存时间表示为$TTL(k)$。当消息$m_k$被某节点产生时,$TTL(k)$的值即被指定。若某条消息的$TTL$到期,则消息会被当前节点丢弃。消息$m_k$的大小用符号$S(k)$表示。模型假设任一节点$n_i$知道其自身接口的带宽$B(i)$。该假设的合理性建立在网络中的每个节点往往在通信时都采用统一标准的某种接口,如蓝牙,wifi等。退一步讲,即使接口状态不稳定,也可以利用滑动窗口法对其平均值进行预测。对于任意一对相遇节点$n_i$及$n_j$,令节点$n_i$记录两个值$t_{s_{(i,j)}}$及$t_{e_{(i,j)}}$,分别用于记录两节点间最近一次连接的开始时间及结束时间。于是$t_{e_{(i,j)}}-t_{s_{(i,j)}}$即代表该次连接的持续时间;为简单起见,符号$t_{s_{(i,j)}}$及$t_{e_{(i,j)}}$在无歧义的上下文中,简记为$t_s$和$t_e$。

PRoPHET路由算法\cite{AndersLindgren2004}记录了节点的相遇历史,用于预测节点间的相遇概率,并且考虑到了相遇概率具有传递性的特点。在PRoPHET中,$P_{(a,b)}$即用于衡量节点$n_a$及$n_b$之间投递概率的效用函数,存于节点$n_a$内。$P_{(a,b)}$的计算及更新方法如下式所示。

\begin{equation}
P_{(a,b)}=P_{(a,b)_{old}}+(1-P_{(a,b)_{old}})\times P_{init}
\label{eq:prophet1}
\end{equation}
\begin{equation}
P_{(a,b)}=P_{(a,b)_{old}}\times\gamma^{k}
\label{eq:prophet2}
\end{equation}
\begin{equation}
P_{(a,c)}=P_{(a,c)_{old}}+(1-P_{(a,c)_{old}})\\
\times P_{(a,b)}\times P_{(b,c)}\times \beta
\label{eq:prophet3}
\end{equation}

其中,$P_{init}$, $\beta$及$\gamma$是位于$[0,1]$范围内的常数。每个节点维护一个$1\times|\mathcal{N}|$的向量,该向量中第$i$个元素记录了节点$n_a$对节点$n_i$的投递率。

\section{问题形式化}
\label{chap4:问题形式化}

本节给出问题的形式化定义。

\subsection{问题目标}
问题目标即是最大化每条消息的投递概率。在机会网络中,每个节点以“存储-携带-转发”的方式对消息进行路由。在机会路由中,一种策略是总是让节点选择对目的节点投递率高的消息进行投递。吞吐量与节点间的接触时间及通信接口带宽具有很强的相关性,因此对消息是否能投递成功具有较大的影响。然而在该种方法中,节点之间连接的吞吐量因素并未考虑进去。由于节点的带宽,接触时间及接触机会非常有限,缓存消息队列的调度策略对路由性能表现具有很大的影响。假设节点$n_a$持有三条消息$m_i$, $m_j$及$m_k$,其目的节点都为$n_b$,大小分别为150 K, 200 K 与100 K。然而当前连接最多只能传输120 K 的数据量。在此情况下,由于连接吞吐量的限制,消息$m_i$和$m_j$都无法被成功传递。一种做法是让$n_a$按消息大小从小到大传递,然而这种方法虽然能保证部分消息成功传输,但并未设定任何优化目标,由此可能使得某些虽然大小稍大,但也能成功传递,且对投递率具有较大改进的消息无法充分利用该次机会传输。为解决此问题,需要对消息的调度设定一个可行的评估指标。在本章中,主要集中研究如何有效的数据项选择机制提高消息的投递率。下面将给出最优化消息传输投递率收益的分析过程。

若消息被节点$n_a$转发给$n_b$,则该消息宣告传输失败,仅当$n_a$及$n_b$都未成功传输该消息,其投递率可以用公式(\ref{eq:probability})计算。

\begin{equation}
P^{\{a,b\}}=1-(1-P_{(a,d_i)})(1-P_{(b,d_i)})
\label{eq:probability}
\end{equation}

于是对于传输收益有如下定义。

\begin{definition}传输收益.\\
传输收益函数$\zeta(i)$用于衡量消息$m_i$在本次传输过程中能够改善的投递率, 有
\begin{equation}
\zeta(i)=P^{\{a,b\}}-P^{\{a\}}\\
=1-(1-P_{(a,d_i)})(1-P_{(b,d_i)})-P_{(a,d_i)}
\label{eq:improve}
\end{equation}
\end{definition}

函数$\zeta(i)$的值代表将消息$m_i$从节点$n_a$传输给节点$n_b$所带来的投递率收益。从这点出发,可以如下定义“吞吐量相关最优机会路由”概念。

\begin{definition} 吞吐量相关最优机会路由. \\
最优机会路由总是尝试最大化消息投递概率,且考虑带宽及接触时间两个因素。节点$n_i$将依照$\zeta$函数值选择消息传输给节点$n_j$,利用预测的吞吐量最大化$\zeta$值总和。
\label{def:throughput-routing}
\end{definition}

例如,在\tablename~\ref{tab:chap4_zeta} 所示例子中,节点$n_a$缓存中总共有5条消息,记为$m_1$, $m_2$, $m_3$, $m_4$, $m_5$。消息$m_i$的目的节点记为$d_i$。对于其中任意一条消息$m_i~(1\leq i\leq 5)$,有$P_{(b,d_i)}>P_{(a,d_i)}$。在这种情形下,当节点$n_a$与$n_b$相遇时,所有这5条消息都应由$n_a$转发给$n_b$。由公式~(\ref{eq:improve})可以得到每条消息对应的$\zeta$值。在下一小节中,将给出吞吐量的预测方法,然后给出问题的形式化定义。

\begin{table}
\centering
  \caption{Five messages in $n_a$'s buffer}
  \label{tab:chap4_zeta}
  \begin{tabular}{cccccc}
  \hline
    data item  & $m_1$ & $m_2$ & $m_3$ & $m_4$ & $m_5$ \\
    \hline
    $P_{(a,d_i)}$ & 0 & 0.2 & 0.1 & 0.12 & 0.2 \\
    $P_{(b,d_i)}$ & 0.1  & 0.75  & 0.2 & 0.25 & 0.35  \\
    value of $\zeta$ & 0.1 & 0.6 & 0.18 & 0.22 & 0.28 \\
    message size & 1K  & 2K  & 5K   & 6K   & 7K   \\
    \hline
  \end{tabular}
\end{table}

\subsection{吞吐量预测}

\begin{definition}连接吞吐量. \\
给定任意两个节点的接触时间 $t$,以及带宽 $B$ (KB/unit) , 该次连接的吞吐量,记为 $TH$ ,有
\[
TH=B\cdot t
\]
\label{def:throughput}
\end{definition}

在此假设带宽$B$为已知,要预测吞吐量,则只需预测出连接的持续时间$t$。采用公式~(\ref{eq:estimate})~预测下一次$n_a$与$n_b$之间连接的持续时间。
\begin{equation}
\tau_{(a,b)_{new}}=(1-\alpha)\tau_{(a,b)_{old}}+\alpha(t_e-t_s)
\label{eq:estimate}
\end{equation}
其中$\alpha\in[0,1]$是常数参数,$t_e-t_s$是节点$n_a$与$n_b$之间最近一次连接的持续时间,该部分加权为$\alpha$。当$\alpha$值设的较大时,公式~(\ref{eq:estimate})后半部分的影响较大,反之亦然。

当节点$n_a$一段时间内没有再与节点$n_b$接触时,则用公式~(\ref{eq:estimate2})更新$\tau_{(a,b)}$。
\begin{equation}
\tau_{(a,b)_{new}}=\tau_{(a,b)_{old}}\gamma^{k}
\label{eq:estimate2}
\end{equation}
其中$\gamma\in[0,1)$是衰减常数,如公式~(\ref{eq:prophet2})一样;$k$是离上一次更新公式流逝的单位时间数目(即经过的时间槽数目)。

在每个时间槽内,对于任意一对节点$n_a$和$n_b$,在每一个时间槽,节点都会检查$n_a$及$n_b$的连接状态。对$\tau$的更新过程如算法~\ref{alg:chap4_estimate}所示。根据定义~\ref{def:throughput}~所示关于$n_a$与$n_b$之间吞吐量的定义,吞吐量可以由公式~(\ref{eq:eth})预测。

\begin{algorithm}[tbp] %算法的开始
\renewcommand{\algorithmicrequire}{\textbf{For}}
\caption{Updating the $\tau$ value} %算法的标题
\label{alg:chap4_estimate} %给算法一个标签,这样方便在文中对算法的引用
\begin{algorithmic}[1] %这个1 表示每一行都显示数字
\REQUIRE the current time unit
\IF{connection is up}
    \STATE $t_s\leftarrow current\_time$
    \STATE $\tau_{(a,b)_{new}}=\tau_{(a,b)_{old}}\gamma^{k}$
    \STATE $k\leftarrow k+1$
\ELSIF{connection is down}
    \STATE $t_e\leftarrow current\_time$
    \STATE $\tau_{(a,b)_{new}}=(1-\alpha)\tau_{(a,b)_{old}}+\alpha(t_e-t_s)$
    \STATE $k\leftarrow 1$
\ELSE
    \STATE $\tau_{(a,b)_{new}}=\tau_{(a,b)_{old}}\gamma^{k}$
    \STATE $k\leftarrow k+1$
\ENDIF    
\end{algorithmic}
\end{algorithm}

\begin{equation}
ETH_{(a,b)}=B_{(a,b)}\cdot\tau_{(a,b)}
\label{eq:eth}
\end{equation}

\subsection{问题定义}

首先证明数据项选择问题可以建模为0-1背包问题,然后给出问题的形式化定义。

\begin{theorem}
\label{thm:knapsack}
定义~\ref{def:throughput-routing}中路由所对应的数据项选择问题,即为0-1背包问题。
\end{theorem}
\begin{proof}
将预测吞吐量$ETH_{(a,b)}$看做背包的最大承重,收益函数值$\zeta(i)$看做第$i$项物品的价值,消息的大小$S(i)$看做第$i$项物品的重量,则数据项选择过程等同于0-1背包问题填装物品的过程,其中决策目标即是让每个物品所对应的收益的总和,即总收益最大。
\end{proof}

由定理~\ref{thm:knapsack}~可知,数据项选择问题是一个$\mathcal{NP}$完全的最优决策问题。该问题形式化定义如下。

\begin{definition} 数据项选择问题.\\
吞吐量相关最优机会路由中的数据项选择问题是一个最优决策问题,即
\[
Max~~~\sum_{i=1}^{n}\zeta(i)x_i
\]\[
s.t.~~~\sum_{i=1}^{n}S(i)x_i\leq ETH_{(a,b)}
\]\[
x_i\in \{0,1\},~~~i=1,\ldots,n
\]
\end{definition}

\section{基于数据项选择的改进路由}
\label{chap4:基于数据项选择的改进路由}

\section{仿真实验}
\label{chap4:仿真实验}

\section{本章小结}
\label{chap4:本章小结}

\chapter{基于跳数的启发式距离向量算法}

本章提出一种用于机会网络的基于跳数的启发式距离向量算法HCH,主要贡献如下:

\begin{enumerate}
\item 设计了启发函数用以预测消息投递所需跳数。启发策略所需的信息由节点间传递的消息数据包携带。
\item 形式化定义了矩阵运算,从而将跳数预测计算转化为矩阵运算。
\item 利用ONE模拟器对HCH算法进行了仿真,结果表明HCH具有较高的投递率以及较低的投递时延,且网络开销保持在可接受水平。
\end{enumerate}

本章组织如下:\ref{chap5:系统模型}节中介绍了系统模型及路由模型;\ref{chap5:消息投递跳数预测}节中提出了跳数预测算法;\ref{chap5:路由协议}节中提出了基于跳数的启发式距离向量路由算法;\ref{chap5:仿真实验}节为仿真实验;\ref{chap5:本章小结}节概括了本章内容。


\section{系统模型}
\label{chap5:系统模型}

在该模型中,网络包含一组移动节点,节点之间对等通信。基本假设如下:
\begin{itemize}
\item 所有节点以对等方式通信,即网络中不存在任何辅助消息进行传递的基础设备。换言之,不存在路由器类似的设备用于转发消息,所有的节点合作以多跳的方式进行消息投递,节点自身将做出对消息的转发决策。
\item 节点的移动方式多变难以预测,即难以对某个节点预测其下一时间或下一时间段的路径及地点。
\end{itemize}

数学符号如\tablename~\ref{tab:chap5_math_table}所示。节点集合以$V={v|1\leq v\leq n}$表示。为便于分析,描述路由的过程只针对某一条消息而言,并以$s$代表其源节点,$d$代表其目的节点,该消息用符号$M_k(s,d)$表示,其索引号为$k$。符号$hop(k)$为一个整数,表述消息$M_k$所经过的跳数值。符号$\overline{hop}(i,j)$表示消息从节点$i$到达节点$j$之间的平均跳数。

在此模型下,本章解决如下问题:
\begin{itemize}
\item 如何设计效用指标衡量网络当前状态
\item 如何从网络中收集信息,从而动态计算效用指标
\item 如何根据效用值选择节点的转发策略
\end{itemize}



\begin{table}[tbp]
  \caption{数学符号定义}
  \label{tab:chap5_math_table}
\centering
  \begin{tabular}{p{0.15\linewidth}<{\centering}p{0.73\linewidth}<{\centering}}
  \hline
   \textbf{notation} & \textbf{meaning}  \\
    \hline
    $n$ & 节点总数\\ 
    $V$ & 节点集合(|V|=n)\\ 
    $M_k(s,d)$ & 索引号为 $k$的消息,其中源节点为$s$,目的节点为$d$\\   
    $\overline{hop}(i,j)$ & 节点 $i$ 和节点$j$之间的平均跳数 \\ 
    $hop(k)$ & 消息 $M_k$ 当前经过的跳数\\
    $h(i,j)$ & 节点 $i$ 和 $j$之间的估计跳数 \\
    \hline
  \end{tabular}
\end{table}


\section{消息投递跳数预测}
\label{chap5:消息投递跳数预测}

\subsection{消息收集}
\label{chap5:消息收集}

\begin{figure}[bt]
  \centering\includegraphics[width=0.7\textwidth]{paper-HCH/matrix}
  \caption{矩阵$\bm{A}$及其对应的滑动窗口}
  \label{fig:chap5_matrix}
\end{figure}

\begin{algorithm}[tbp] %算法的开始
\caption{Maintaining the matrix $\boldmath{A}$ and its slide-windows} %算法的标题
\label{alg:chap5_matrix} %给算法一个标签,这样方便在文中对算法的引用
\begin{algorithmic}[1] %这个1 表示每一行都显示数字
\REQUIRE  %算法的输入参数:Input
packet $M_k$, current time $t+1$
\ENSURE  %算法的输出:Output
matrix $\boldmath{A}$, slide-window $win$ \\
When packet $M_k$ comes\\
\textbf{local variables:} $i,j,c$
\STATE $i\leftarrow M_k.getSourceNodeID()$
\STATE $Sequence\leftarrow M_k.getPassedNodes()$
\STATE $c\leftarrow 1$
\FOR{$j\in Sequence$}
    \STATE $win[i,j,t]\leftarrow c$
    \STATE $c\leftarrow c+1$
\ENDFOR
\FOR{$i\leftarrow 1$ to $n$}
    \FOR{$j\leftarrow 1$ to $n$}
        \STATE $a_{i,j}\leftarrow \left(\sum_{k=t-r+1}^{k=t}win[i,j,k]\middle)\right/r$
    \ENDFOR
\ENDFOR
\RETURN $\boldmath{A}$, $win$ %算法的返回值
\end{algorithmic}
\end{algorithm}

\subsection{启发函数}
\label{chap5:启发函数}

\begin{equation}
\label{eq:H}
\mathcal{H}(i,k) = hop(k) + h(i, d)
\end{equation}

\begin{algorithm}[tbp] %算法的开始
\caption{Heuristic value calculation} %算法的标题
\label{alg:chap5_heuristic} %给算法一个标签,这样方便在文中对算法的引用
\begin{algorithmic}[1] %这个1 表示每一行都显示数字
\REQUIRE  %算法的输入参数:Input
Matrix $\boldmath{A}$ of node $i$
\ENSURE  %算法的输出:Output
$h(i,*)$\\
\textbf{local variables:} $i,h,c,d$ \\
\FOR{$M_k\in i.messages$}
    \STATE $h\leftarrow 0$
    \STATE $c\leftarrow 0$
    \STATE $M\leftarrow \Lambda$
    \STATE $i\leftarrow getHostID()$
    \STATE $d\leftarrow M_k.getDestinationID()$
    \REPEAT
        \STATE $M\leftarrow M\bigodot A$
        \STATE $h\leftarrow h+m_{i,d}$
        \STATE $c\leftarrow c+1$
    \UNTIL{$m_{i,d}=0$}
    \STATE $h(i,d)=h/c$
\ENDFOR
\RETURN $h(i,*)$ %算法的返回值
\end{algorithmic}
\end{algorithm}

\begin{figure}[bt]
  \centering
  \includegraphics[width=0.6\textwidth]{paper-HCH/heuristic}
  \caption{跳数计算过程}
  \label{fig:heuristic}
\end{figure}

\section{路由协议}
\label{chap5:路由协议}

\begin{table}[bt]
  \caption{机会路由决策}
  \label{tab:chap5_routing}
  \centering
  \begin{tabular}{cc}
  \hline
   \textbf{策略:持有该消息的节点} & \textbf{对应情况}  \\
    \hline
    节点v和节点u & 向网络中增加一个新的消息副本\\
    节点v & 节点u不如节点v优\\
    u & 节点u比节点v优\\
    \hline
  \end{tabular}
\end{table}

\section{仿真实验}
\label{chap5:仿真实验}

\begin{table}
\centering
\caption{Helsinki City场景仿真设置}
\label{tab:simulation_helsinki}
\begin{tabular}{
p{0.45\linewidth}<{\centering}
p{0.5\linewidth}<{\centering}
}
\hline
\textbf{parameter name} & \textbf{range(default value)} \\
\hline
number of nodes & 120  \\
world size($m\times m$) & 4500$\times$3000  \\
tickets for S \& W & 13 \\
message TTL(min) & 200--500 (300) \\
simulation time(hours) & 12 \\
message size(KB) & 500--1024 \\
pedestrian buffer(MB) & 15--55 (15) \\
tram buffer(MB) & 500 \\
bluetooth range(m) & 10 \\
highspeed range(m) & 1000 \\ 
bluetooth bandwidth(KBps) & 250 \\
highspeed bandwidth(MBps) & 10 \\ 
pedestrian speed(m/s) & 0.5--1.5  \\
message interval(s) & 35--40 \\
\hline
\end{tabular}
\end{table}

\section{本章小结}
\label{chap5:本章小结}
%%==================================================
%% conclusion.tex for SJTU Master Thesis
%% based on CASthesis
%% modified by wei.jianwen@gmail.com
%% version: 0.3a
%% Encoding: UTF-8
%% last update: Dec 5th, 2010
%%==================================================

\chapter{总结及展望}
%\chapter*{总结及展望\markboth{总结及展望}{}}
%\addcontentsline{toc}{chapter}{全文总结}

目前,针对容迟网络的研究主要集中在路由协议领域,如何做出正确高效的路由选择是无线网络领域内的关键技术和主要研究课题。通过对近年来的主要研究成果进行分析,移动社会网络报文传输机制的研究是容迟网络研究在引入社会网络分析等技术后的最新发展趋势。最新的移动社会网络信息共享研究也是强调在容迟网络分布式体系结构下高容量、低成本的数据传输。本文基于移动模式最优节点群组选取算法,建立了周期相关的移动记录模型,并从移动记录中提取出节点(群组)移动模式。对于成组的节点,其被看做一个整体,并评估整体的预测投递率。本文对两个相关路由的关键属性进行研究分析,并将路由问题建模为组合最优化问题,且证明了该问题的$\mathcal{NP}$难解性。为求解该路由问题,基于局部搜索及禁忌搜索,分别提出了两种路由算法Local-MPAR及Tabu-MPAR。此外,证明了Tabu-MPAR过程可以使得持有消息的节点集合最终达到预测投递概率最优的集合。仿真实验表明:两种MPAR算法,在机会网络中的综合移动模型WDM上优于DF算法及SimBet算法。

本文基于消息传输收益的最优队列调度算法,针对节点间带宽和接触时间受限所导致的连接有限的吞吐量会造成消息的传输失败,从而浪费了节点间宝贵的接触机会的现象,通过定义概念“消息传输效用”,在PRoPHET的基础上加上数据项选择机制,得到改进后的路由算法Throughput。仿真实验表明:该数据项选择机制能够在消息投递率,消息投递时延,网络开销三方面较大程度的改进路由性能表现。

最后,本文利用启发函数,基于跳数信息进行消息投递跳数预测。利用滑动窗口机制,可以动态更新记录平均跳数的矩阵。基于此,定义了启发式函数用于预测当前节点到目的节点之间的潜在跳数。为了便于计算,定义了一种矩阵运算符,从而将跳数的估计过程转化为矩阵运算。仿真实验表明:本文提出的HCH算法在综合性能表现上优于Epidemic, S \& W以及PRoPHET三种算法。

智能交通系统所依赖的一种组网方式,即车联网,亦具有延迟容忍的特性,最新的研究趋势则是从延迟容忍网络及移动社会网络的角度,对车联网的路由问题进行研究。本文提出的基于移动模式的节点群组选取算法及对应的路由协议,正可用于社会属性相关的一类网络中。未来的研究将可集中于车联网的环境,针对车联网中既具有社会属性亦具有周期移动特性的一类节点进行分析,建立出能够如实反应其移动模式的数学模型,从而实现最优路由。此外,消息副本的分发速度是衡量路由性能的很重要的指标,本文对网络中消息副本分发速度只从仿真实验进行了验证,而未从理论上保证其可靠性。在以后的工作中,拟建立数学模型,对消息副本的分发过程进行分析,从理论上推导消息副本数量与分发时延之间的关系,从而保证消息的投递时延。

 %% 全文总结



%%%%%%%%%%%%%%%%%%%%%%%%%%%%%% 
%% 文后(无章节编号)
%%%%%%%%%%%%%%%%%%%%%%%%%%%%%% 
\backmatter

% 参考文献
% 使用 BibTeX
% 包含参考文献文件.bib
%\bibliography{reference/chap1,reference/chap2}
\bibliographystyle{ieeetr}
\bibliography{/Users/lei/Dropbox/papers.bib}

%% 个人简历(硕士学位论文没有个人简历要求)
% \include{body/resume}

\listoftables
\listoffigures

%\renewcommand{\listalgorithmcfname}{算法索引}
\listofalgorithms

% 发表文章目录
%%==================================================
%% pub.tex for SJTU Master Thesis
%% based on CASthesis
%% modified by wei.jianwen@gmail.com
%% version: 0.3a
%% Encoding: UTF-8
%% last update: Dec 5th, 2010
%%==================================================



\begin{publications}{99}

%    \item\textsc{Chen H, Chan C~T}. {Acoustic cloaking in three dimensions using acoustic metamaterials}[J].
%      Applied Physics Letters, 2007, 91:183518.
%
%    \item\textsc{Chen H, Wu B~I, Zhang B}, et al. {Electromagnetic Wave Interactions with a Metamaterial Cloak}[J].
%      Physical Review Letters, 2007, 99(6):63903.

\item\textsc{\textbf{L.~You}, J.~Li, C.~Wei, L.~Hu}, ``MPAR: A Movement Pattern-Aware Optimal Routing for Social Delay Tolerant Networks,'' in \textit{Ad Hoc Networks} (\textbf{SCI}, Impact Factor \textbf{1.943}), Vol.24, 2015, pp. 228-249.

%\item MPAR: A Movement Pattern-Aware Optimal Routing for Social Delay Tolerant Networks, in \textit{Ad Hoc Networks} (\textbf{SCI}, Impact Factor \textbf{1.943}), Vol.24, 2015, pp. 228-249.

%
\item\textsc{\textbf{L.~You}, L.~Lei, D.~Yuan}, ``Range Assignment for Power Optimization in Load-Coupled Heterogeneous Networks,'' \textit{IEEE International Conference on Communication Systems (IEEE ICCS)}, 2014.

%\item Range Assignment for Power Optimization in Load-Coupled Heterogeneous Networks, \textit{IEEE International Conference on Communication Systems (IEEE ICCS) 2014}, accepted

%
\item\textsc{\textbf{L.~You}, L.~Lei, D.~Yuan}, ``A Performance Study of Energy Minimization for Interleaved and Localized FDMA,'' \textit{IEEE International Workshop on Computer-Aided Modeling Analysis and Design of Communication Links and Networks (IEEE CAMAD)}, 2014

%\item A Performance Study of Energy Minimization for Interleaved and Localized FDMA, \textit{IEEE International Workshop on Computer-Aided Modeling Analysis and Design of Communication Links and Networks (IEEE CAMAD) 2014}, accepted


%
%
\item\textsc{\textbf{L.~You}, J.~Li, C.~Wei, C.~Dai}, ``A One-hop Information Based Geographic Routing Protocol for Delay Tolerant MANETs,'' in \textit{International Journal of Ad Hoc and Ubiquitous Computing} (\textbf{SCI}, Impact Factor \textbf{0.900}), in press.

%\item A One-hop Information Based Geographic Routing Protocol for Delay Tolerant MANETs, in \textit{International Journal of Ad Hoc and Ubiquitous Computing} (\textbf{SCI}, Impact Factor \textbf{0.900}), in press.


%
\item\textsc{\textbf{L.~You}, J.~Li, C.~Wei, C.~Dai, J.~Xu}, ``A General and Specific Utility Based Adaptive Routing for Delay Tolerant Networks,'' in \textit{International Journal of Distributed Sensor Networks} (\textbf{SCI}, Impact Factor \textbf{0.923}), Vol.2014, 15 pages, 2014.

%\item A General and Specific Utility Based Adaptive Routing for Delay Tolerant Networks, in \textit{International Journal of Distributed Sensor Networks} (\textbf{SCI}, Impact Factor \textbf{0.923}), Vol.2014, 15 pages, 2014.

%
\item\textsc{\textbf{L.~You}, J.~Li, C.~Wei, J.~Xu, C.~Dai}, ``A Data Item Selection Mechanism for Mobile Opportunistic Networks,'' in \textit{International Journal of Distributed Sensor Networks} (\textbf{SCI}, Impact Factor \textbf{0.923}), Vol.2014, 11 pages, 2014.

%\item A Data Item Selection Mechanism for Mobile Opportunistic Networks, in \textit{International Journal of Distributed Sensor Networks} (\textbf{SCI}, Impact Factor \textbf{0.923}), Vol.2014, 11 pages, 2014.

%
\item\textsc{\textbf{L.~You}, J.~Li, C.~Wei, C.~Dai, J.~Xu, L.~Hu}, ``A Hop Count Based Heuristic Routing Protocol for Mobile Delay Tolerant Networks'', in \textit{The Scientific World Journal} (\textbf{SCI}, Impact Factor \textbf{1.219}), Vol.2014, 12 pages, 2014.

%\item A Hop Count Based Heuristic Routing Protocol for Mobile Delay Tolerant Networks, in \textit{The Scientific World Journal} (\textbf{SCI}, Impact Factor \textbf{1.219}), Vol.2014, 12 pages, 2014.

%
\item\textsc{\textbf{L.~You}, J.~Li, S.~Jiang, C.~Dai}, ``A Hop Count Based Multi-path Forwarding Routing for Delay Tolerant Networks'', {\it Advances in Information Sciences and Service Sciences}, 5(7), pp.1199-1207, 2013.

%\item A Hop Count Based Multi-path Forwarding Routing for Delay Tolerant Networks, in {\it Advances in Information Sciences and Service Sciences}, 5(7), pp.1199-1207, 2013.

\item 《面向社交信息分享的移动容迟网络消息传输研究》,山东省研究生优秀科技创新成果三等奖
    
\end{publications}




%% 参与项目列表
%\include{body/projects}

%%%%%%%%%%%%%%%%%%%%%%%%%%%%%% 
%% 附录(章节编号重新计算,使用字母进行编号)
%%%%%%%%%%%%%%%%%%%%%%%%%%%%%% 
\appendix

% 附录中编号形式是"A-1"的样子
\renewcommand\theequation{\Alph{chapter}--\arabic{equation}}
\renewcommand\thefigure{\Alph{chapter}--\arabic{figure}}
\renewcommand\thetable{\Alph{chapter}--\arabic{table}}

%\include{body/app1} % 更新记录
%\include{body/app2} % 麦克斯韦方程
% \include{body/app3}

%% 插图索引
%\listoffigures
%\addcontentsline{toc}{chapter}{\listfigurename} %将图索引加入全文目录
%% 表格索引
%\listoftables
%\addcontentsline{toc}{chapter}{\listtablename}  %将表格索引加入全文目录
%
%% 主要符号、缩略词对照表
%\include{body/symbol}


% 致谢
%%==================================================
%% thanks.tex for SJTU Master Thesis
%% based on CASthesis
%% modified by wei.jianwen@gmail.com
%% version: 0.3a
%% Encoding: UTF-8
%% last update: Dec 5th, 2010
%%==================================================

\begin{thanks}

强烈感谢我的论文指导老师魏长江教授和李建波副教授,他们对我进行了无私的指导和帮助,不厌其烦的帮助进行论文的修改和改进。感谢我的女友胡乐娟在我科研最困难的过程中给予我的关心和照顾,没有她的帮助,我很难想象自己如何支撑到科研工作最终完成那一刻。感谢在做论文的过程中,帮助我调整文档格式的各位同学们,他们在我写论文的过程中给予我了很多重要素材,还在论文的撰写 和排版过程中提供热情的帮助。另外,在校图书馆查找资料的时候,图书馆的老师也给我提供了很多方面的支持与 帮助。在此向帮助过我的老师致以真诚的谢意!

\end{thanks}


% 论文原创性声明和使用授权
\makeDeclareOriginal

\end{document}
